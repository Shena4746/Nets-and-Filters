\documentclass{report}
%\documentclass[a4paper,12pt]{article}
\usepackage{mystyle}
\usepackage{mypackages}
\usepackage{commands}
\usepackage[T1]{fontenc}
\usepackage{lmodern}
\mathtoolsset{showonlyrefs=true}

\begin{document}

\section{Net and Filter as Set-Theoretic Objects}

\subsection{Overview}

\subsection{Filter}
\subsubsection{Definition}
\begin{dfn} (Filter)

    A nonempty collection \( \mathscr{F} \) of subsets of a set \( X \) is called a filter of \( X \) if it satisfies
    \begin{itemize}
        \item \( \emptyset \notin \mathscr{F} \).
        \item \( X \supset B \supset A \in \mathscr{F} \implies B \in \mathscr{F}\).
        \item \( A, B \in \mathscr{F} \implies A \cap B \in \mathscr{F}\).
    \end{itemize}

    Given two filters \( \mathscr{F} \) and \( \mathscr{G} \) of the same set, \( \mathscr{G} \) is called finer than \( \mathscr{F} \), or \( \mathscr{F} \) is called coarser than \( \mathscr{G} \), if \( \mathscr{G} \supset \mathscr{F}\), and said to be strictly so if the inclusion is strict.
\end{dfn}

\begin{rem}
    Every filter \( \mathscr{F} \) has the following finite intersection property; for every finitely many member \( F_1 ,\ldots, F_n \) of \( \mathscr{F} \) there holds \( \cap F_i \neq \emptyset \).
\end{rem}

As is often the case with other concepts in mathematics, a filter can be generated by its smaller subset.
\begin{dfn} (Filter basis)
    A nonempty collection \( \mathscr{B} \) of subset of a set \( X \) is called a filter basis if it has the following property;
    \begin{itemize}
        \item \( \emptyset \notin \mathscr{B} \).
        \item For \( B_1, B_2 \in \mathscr{B} \), there is \( B_3 \in \mathscr{B} \) such that \( B_3 \subset B_1 \cap B_2 \).
    \end{itemize}

    A filter basis \( \mathscr{B} \) is called a basis of a filter \( \mathscr{F}  \) if for every \( F \in \mathscr{F} \) there is \( B \in \mathscr{B} \) such that \( B \subset F \).
\end{dfn}

\begin{rem}
    Given a filter basis \( \mathscr{B} \), the filter \( \mathscr{F} \) defined by
    \begin{equation*}
        \mathscr{F} = \{F \mid F \supset B,\,\exists B \in \mathscr{B}\}
    \end{equation*}
    contains \( \mathscr{B} \) as it subcollection. It is easy to see that \( \mathscr{B} \) is a basis of \( \mathscr{F} \). Whence \( \mathscr{B} \) is said to generate \( \mathscr{F} \).
\end{rem}

A typical set theoretic argument establishes the following minimality property.
\begin{prp}\label{coarsest generation}
    The filter generated by a filter basis is the coarsest filter containing the basis.
\end{prp}

\begin{ex} (Filter generated by a set)
    \( \mathscr{B}:=\{A\} \) with \( A \neq \emptyset \) is the simplest filter basis. The filter generated by \( \mathscr{B} \) is called the filter generated by \( A \), and denoted by \( \left< A \right> \).
\end{ex}


\begin{dfn} (Trace of filter basis)
    A filter basis is said to have trace on a set \( A \) if \( A \) intersects every member of the filter basis.
\end{dfn}

\begin{dfn} (Compatibility)
    Two filter bases \( \mathscr{A} \) and \( \mathscr{B} \) are said to be compatible if \( A \cap B \neq \emptyset \) for every \( A \in \mathscr{A} \) and \( B \in \mathscr{B} \).
\end{dfn}

\begin{prp}\label{compatible} (Characterization of compatibility)
    Two filter bases \( \mathscr{F} \) and \( \mathscr{G} \) are compatible if and only if they admits a common finer filter.
\end{prp}
\begin{prf}
    If they are compatible, then
    \begin{equation*}
        \{F \cap G \mid F \in \mathscr{F},\,G \in \mathscr{G}\}
    \end{equation*}
    is a common finer filter.

    Conversely, suppose a common finer filter exists. Then every \( F \in \mathscr{F} \) and \( G \in \mathscr{G} \) is a member of the filter, and thus \( F \cap G \neq \emptyset \) necessarily follows from definition of filter.
    \qed\end{prf}

\begin{rem}\label{rem trace}
    If a filter \( \mathscr{F} \) has trace on \( A \), then we can construct a finer filter since the filter \( \left< A \right> \) generated by \( A \) is compatible with \( \mathscr{F} \). This procedure yields a strictly finer filter if and only if \( A \notin \mathscr{F}\). This fact is frequently exploited.
\end{rem}

\subsubsection{Transformation of Filter}

\begin{dfn} (Image of filter)
    Let \( f:X \to Y \) be a map.
    The image of a filter \( \mathscr{F} \) of \( X \) under \( f \) is the filter generated by the filter basis
    \begin{equation*}
        I_f(\mathscr{F}):=\{f(F) \mid F \in \mathscr{F}\},
    \end{equation*}
    and is denoted by \( f(\mathscr{F}) \). The image of a filter basis \( \mathscr{B} \) is defined to be \( I_f(\mathscr{B}) \) itself, and is also denoted by \( f(\mathscr{B}) \).
\end{dfn}

\begin{exe}
    Let \( f:X \to Y \) be a map, and let \( \mathscr{F} \) be a filter of \( X \). If \( f \) is surjective, then \( f(\mathscr{F}) = I_f(\mathscr{F}) = \{G \subset Y \mid f^{-1}(G) \in \mathscr{F}\} \).
\end{exe}

\begin{exe}
    Let \( f:X \to Y \) be a map, and let \( \mathscr{A} \) and \( \mathscr{B} \) be filter bases of \( X \). Then \( \mathscr{A} \subset \mathscr{B} \) implies \( f(\mathscr{A}) \subset f(\mathscr{B}) \), and conversely if \( f \) is injective. (Use the fact that \( f^{-1}(f(A)) =A \) if \( f \) is injective.)
\end{exe}

\begin{exe}\label{inverse image of filter} (Inverse image of filter basis)
    Let \( f:X \to Y \) be a map, and let \( \mathscr{B} \) be a filter basis of \( Y \). Then \( f^{-1}(\mathscr{B}):=\{f^{-1}(B) \mid B \in \mathscr{B}\} \) is a filter basis of \( X \) if and only if \( \mathscr{B} \) has trace on \( f(X) \).

    When one of these conditions are fulfilled, \( f^{-1}(\mathscr{B}) \) is called the inverse image of \( \mathscr{B} \) under \( f \). We  use the same notation for the filter generated by the filter basis \( f^{-1}(\mathscr{B}) \).
\end{exe}

\begin{exe}\label{double mapping}
    Let \( f:X \to Y \) be a map. Suppose \( \mathscr{B}_X \) and \( \mathscr{B}_Y \) are filter bases of \( X \) and \( Y \), respectively.
    Then the following relation holds, provided that each of the inverse image is properly defined as a filter basis.
    \begin{itemize}
        \item[(1)] \( f(f^{-1}(\mathscr{B}_Y)) \supset \mathscr{B}_Y \).
        \item[(2)] \( f^{-1}(f(\mathscr{B}_X)) \subset  \mathscr{B}_X \), with equality if \( f \) is injective.
    \end{itemize}
\end{exe}

\begin{exe}\label{inverse image preserve order}
    Let \( f:X \to Y \) be a surjective map, and let \( \mathscr{A} \) and \( \mathscr{B} \) be filter bases of \( Y \). Then \( \mathscr{A} \subset \mathscr{B} \) if and only if \( f^{-1}(\mathscr{A}) \subset f^{-1}(\mathscr{B}) \), where the inverse images are considered as filter bases. (Use the fact that \( f(f^{-1}(A)) =A \) if \( f \) is surjective.)
\end{exe}

% An important example of compatible classes is the convergent filter bases.
% \begin{prp}
% 	If each of two filter basis \( \mathscr{A} \) and \( \mathscr{B} \) converges to the same point, then they are compatible.
% \end{prp}
% \begin{prf}
% 	Suppose they converge to a point \( p \), and suppose they generate \( \mathscr{F} \) and \( \mathscr{G} \).
% 	It is easy to see that the generated filters also converge to \( p \).

% 	Since the relation \( F \cap G = \emptyset \) for some \( F \in \mathscr{F} \) and \( G \in \mathscr{G}\) means one of these filters cannot converge to \( p \), they must be compatible. Thus, the given filter bases are also compatible by definition of compatibility.
% 	\qed\end{prf}

\subsubsection{Induced Filter}

\begin{dfn} (Induced filter)
    Suppose \( X \) is a set, and suppose \( (X_i)_{i \in I} \) is an indexed set. Let \( \mathscr{F}_i \) be a filter on \( X_i \) and \( \pi_i:X \to X_i \) a map such that \( \pi_i ^{-1}(\mathscr{F}_i) \) is a filter basis of \( X \) for every \( i \in I \). Then the filter \( \mathscr{F} \) induced on \( X \) by \( (\mathscr{F}_i ,\pi_i)_{i \in I} \) is the coarsest one satisfying
    \begin{equation*}
        \pi_i(\mathscr{F}) = \mathscr{F}_i
    \end{equation*}
    for every \( i \in I \).
\end{dfn}

\begin{ex} (Product filter)
    In the above definition, take \( \pi_i :X \ni x = (x_i) \mapsto x_i \in X_i \) as the natural projection. The resulting filter is called the \textit{product filter} of \( \mathscr{F}_i \), and is denoted by \( \prod \mathscr{F}_i \). Note that each \( \pi_i \) is surjective, and therefore the inverse image of filter is automatically well-defined.
\end{ex}

\begin{ex} (Restriction of filter onto trace)
    Suppose a filter \( \mathscr{F} \) of \( X \) has trace on \( A \). Then the pair of \( \mathscr{F} \) and an inclusion map \( i_A:A \to X \) fulfills the condition of Exercise \ref{inverse image of filter}. The resulting induced filter of \( A \) given by \(i_A^{-1}(\mathscr{F}) =\{F \cap A \mid F \in \mathscr{F}\} \) is called the \textit{restriction} of \( \mathscr{F} \) onto \( A \).
\end{ex}

\begin{prp}\label{characterize induced filter} (Characterization of induced filter)
    Let \( \mathscr{F} \) be a filter of \( X \) induced by the pair \( (\mathscr{F}_i ,\pi_i)_{i \in I} \) as in the above definition. Then \( \mathscr{F} \) is generated by a filter basis of \( X \) consisting of all subsets of the form \( \cap_{i \in J} \pi_i ^{-1}(F_i)\), where \( J \) is a finite subset of \( I \) and \( F_i \in \mathscr{F}_i \).
\end{prp}
\begin{prf}
    Let \( \mathscr{B} \) be the filter basis indicated in the claim, and let \( \mathscr{G} \) be the filter generated by \( \mathscr{B} \). It suffice to show that \( \mathscr{F}=\mathscr{G} \).

    Suppose \( A \in \mathscr{F}\). It follows from definition that, for each \( i \), there holds \( \pi_i(A) = F_i \) for some \( F_i \in \mathscr{F}_i \), and hence \( A \supset \pi_i ^{-1}(F_i) \), in which the last term is a member of the basis of \( \mathscr{G} \). Thus, \( \mathscr{F} \subset \mathscr{G} \). To prove the opposite inclusion, we claim \( \pi_i(\mathscr{B})\subset \mathscr{F}_i \) for every index \( i \), from which we deduce that \( \pi_i(\mathscr{G})\subset \mathscr{F}_i \) for every index \( i \), and thus \( \mathscr{G} \subset \mathscr{F} \) by minimality of \( \mathscr{G} \). Suppose \( A \in \mathscr{B}\), that is, suppose \( A \) is of the form
    \[
        A = \bigcap_{j \in J} \pi_j^{-1}(F_j),
    \]
    where \( J \) is a finite subset of \( I \), and \( F_j \in \mathscr{F}_j \). Then we have
    \[
        \pi_j(A) \subset \bigcap_{j \in J}F_j \subset F_j \in \mathscr{F}_j
    \]
    for every \( j \in J \), from which our claim follows.
    \qed\end{prf}

\subsubsection{Maximal Filter}

\begin{dfn} (Maximal, ultra-filter)
    A filter \( \mathscr{F} \) of \( X \) is called a maximal (or ultra) filter if it admits no strictly finer filter of \( X \).
\end{dfn}

\begin{prp}\label{existence maximal filter} (Existence of maximal filter)
    If a collection of sets has the finite intersection property, then there exists a maximal filter containing it.
\end{prp}
\begin{prf}
    Let \( \mathscr{A} \) be the collection of sets.
    Let \( \mathscr{C} \) be the collection of all collections of sets such that each \( \mathscr{F}_i \in \mathscr{C} \) has the finite intersection property and \( \mathscr{F}_i \supset \mathscr{A} \). Introduce an order on \( \mathscr{C} \) by the rule \( \mathscr{F}_1 \succeq \mathscr{F}_2 \iff \mathscr{F}_1 \supset \mathscr{F}_2\).
    It is easy to see that \( \mathscr{C} \) is partially ordered, and that every totally ordered subset admits a supremum. We then have a maximal element \( \mathscr{F} \) of \( \mathscr{C} \) by Zorn's Lemma.
    It remains to show that \( \mathscr{F} \) so derived is a filter.

    \( \emptyset \notin \mathscr{F} \) is obvious.
    For \( B \supset A \in \mathscr{F} \), we see that \( \{B\}\cap \mathscr{F} \in \mathscr{C} \) and hence \( B \in \mathscr{F} \) by maximality.
    For \( A, B \in \mathscr{F} \), we similarly have \( \{A \cap B\} \cup \mathscr{F} \in \mathscr{C} \) by the finite intersection property, and thus \( A \cap B \in \mathscr{F} \) again by maximality.
    \qed\end{prf}

% \begin{rem}
% 	The proposition just stated provides easy access to an underlying maximal filter. For convenience, we sometimes prefer to working with a maximal filter instead of a non-maximal one, provided of course that the property of interact is invariant under this switch.
% \end{rem}

\begin{thm}\label{characterize maximal filter} (Characterization of maximal filter)
    Let \( \mathscr{F} \) be a filter in a set \( X \). The following conditions are equivalent.
    \begin{itemize}
        \item[(1)] \( \mathscr{F} \) is maximal.
        \item[(2)] Every set intersecting every member of \( \mathscr{F} \) belongs to \( \mathscr{F} \).
        \item[(3)] For any subset \( A \) of \( X \), either \( A \) or \( X \setminus A \) belongs to \( \mathscr{F} \).
    \end{itemize}
\end{thm}
\begin{prf}
    (1) \( \implies \) (2):
    Let \( A \) be such intersecting subset and let \( \mathscr{F} \) be maximal. Let \( \mathscr{C} \) be the collection of sets consisting of the sets \( C \) such that \( C \supset A \cap B \) for some \( B \in \mathscr{F} \). It is easy to check that \( \mathscr{C} \) is a filter, and that there holds \( \mathscr{F} \subset \mathscr{C} \) and \( A \in \mathscr{C} \). It then follows from maximality that \( \mathscr{F} = \mathscr{C} \), and hence \( A \in \mathscr{F} \).

    (2) \( \implies \) (3):
    Suppose (3) fails, that is, suppose \( A \notin \mathscr{F} \) and \( X \setminus A \in \mathscr{F} \) for some \( A \subset X\). It necessarily follows that \( F \nsubseteq A \) and \( F \nsubseteq X \setminus A \) for every \( F \in \mathscr{F} \). Then \( A \cap F \neq \emptyset \) and \( \left( X \setminus A \right) \cap F \neq \emptyset \) for every \( F \in \mathscr{F} \). Thus, (2) fails.

    (3) \( \implies \) (1):
    Suppose \( A \in \mathscr{G} \supset \mathscr{F} \), where \( \mathscr{G} \) is a filter. It necessarily follows from the definition of filter that \( X \setminus A \notin \mathscr{G} \), and hence \( X \setminus A \notin \mathscr{F} \). Thus, \( A \in \mathscr{F} \) by Assumption.
    \qed\end{prf}

\begin{prp}\label{invariace of maximality under surjection} (Invariance of filter maximality under surjection)
    Let \( f:X \to Y \) be a surjective map. If a filter \( \mathscr{F} \) of \( X \) is maximal, so is \( f(\mathscr{F}) \).
\end{prp}
\begin{prf}
    Let \( G \subset Y \). By maximality, there holds either \( f^{-1}(G)\in \mathscr{F} \) or not. If this is the case, it follows that \( f^{-1}(G) = F \) for some \( F \in \mathscr{F} \), and hence \( G \supset f(f^{-1}(G))=f(F) \), implying \( G \in f(\mathscr{F}) \). If not, we have \( X \setminus f^{-1}(G) = f^{-1}(X \setminus G) \in \mathscr{F} \). Thus, \( X \setminus G \in f(\mathscr{F}) \).
    \qed\end{prf}

\begin{ex} (Counterexample to the converse)
    Here we see how the converse of Proposition \ref{invariace of maximality under surjection} fails. Let \( \mathscr{N} \) be the set of open intervals over 0 in real line:
    \[
        \mathscr{N} := \{(a,b) \subset \mathbb{R} \mid a<0<b\}.
    \]
    Observe that, by Proposition \ref{characterize maximal filter}, \( \mathscr{N} \) is not maximal while \( \mathscr{N}\cup \{0\} \) is.
    Define \( f:\mathbb{R}\to \mathbb{R}_+ \) via
    \[
        f(x):=\max \{|x|-1,0\}.
    \]
    Then we have
    \[
        f(\mathscr{N}) =\{[0,c) \subset \mathbb{R}_+ \mid c>0\} \cup \{0\} = f(\mathscr{N}\cup \{0\}).
    \]
    Proposition \ref{invariace of maximality under surjection} shows that \( f(\mathscr{N}) = f(\mathscr{N}\cup \{0\}) \) is maximal.
\end{ex}

\begin{prp} (Representation of filter)
    Every filter is the intersection of all finer maximal filter.
\end{prp}
\begin{prf}
    Suppose a filter \( \mathscr{F} \) is given. It is obvious that the intersection is finer than \( \mathscr{F} \). Conversely, suppose \( A \notin \mathscr{F}\). Then \( \mathscr{F} \) has trance on \( X \setminus A \) since \( A \) contains no members of \( \mathscr{F} \), from which it follows that \( \mathscr{F} \) admits a finer filter that contains \( X \setminus A \). Thus, every maximal filter finer than \( \mathscr{F} \) necessarily contains \( X \setminus A \).
    \qed\end{prf}


\subsection{Net}
\begin{dfn} (Net)
    Let \( \Lambda \) be a directed set and let \( X \) be a set.
    A net (on \( \Lambda \)) is a mapping \( x: \Lambda \ni \lambda \mapsto x_{\lambda}\in X \), and is denoted by \( (x_{\lambda})_{\lambda \in \Lambda} \) or \( x_{\lambda} \), or sometimes simply by \( x \) if there is no fear of confusion.
\end{dfn}

\begin{dfn} (Eventually and frequently)
    Let \( (x_{\lambda})_{\lambda \in \Lambda} \) be a net of a directed set \( \Lambda \) into a set \( X \). Suppose \( A \) is a subset of \( X \).
    \begin{itemize}
        \item \( x \) is said to be eventually in \( A \) or residual in \( A \) if there is \( \lambda_0 \in \Lambda \) such that \( x_{\lambda} \in A\) for all \( \lambda >\lambda_0 \).
        \item \( x \) is said to be frequently in \( A \) or cofinal in \( A \) if for every \( \lambda_0 \in \Lambda \) there is \( \lambda > \lambda_0\) such that \( x_{\lambda} \in A\).
        \item \( x_{\lambda} \) is called a maximal net or an ultra-net if \( x \) is eventually either in \( A \) or \( X \setminus A \) for every \( A \subset X \).
    \end{itemize}
\end{dfn}


\begin{dfn} (Subnet, \( \succeq \))
    Let \( x_{\lambda} \) be a net in a set \( X \). We say that a net \( \left( s_d \right)_{d \in D} \) is a subnet of \( x_{\lambda} \), and write \( s \succeq x \), if for every \( \lambda \in \Lambda \) there is \( d \in D \) such that
    \begin{equation*}
        s(D_d) \subset x(\Lambda_{\lambda}),
    \end{equation*}
    where \( D_d:=\{ e \in D \mid e >d \} \) and \( \Lambda_{\lambda} \) is defined similarly.

    If \( s \) is a subnet of a net \( x \) but not the converse, then \( s \) is called a proper subnet of \( x \).
\end{dfn}

\begin{lem}\label{characterize subnet}
    A net \( g \) is a subnet of a net \( f \) if and only if \( g \) is eventually in \( A \) whenever \( f \) is eventually in \( A \).
\end{lem}

\begin{dfn} (Relation of subnet: \( \sim \))
    Two nets \( x \) and \( y \) are called equivalent, and are denoted by \( x \sim y \), if they are subnets of each other, that is, if \( x \succeq y \) and \( y \succeq x \).
\end{dfn}

\begin{rem}
    It is obvious that the relation \( \succeq \) and the equivalence relation \( \sim \) fulfill the axiom of partial order and the axiom of equivalence relation, respectively.
\end{rem}

\end{document}