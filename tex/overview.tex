\documentclass[a4paper,12pt]{article}
\usepackage{mystyle}

\begin{document}
\addcontentsline{toc}{section}{Overview}
\section*{Overview}
This is a personal note on nets and filters, derived from a study on functional analysis.
The main purpose of this document is to provide a short introduction to the study of well-known topological concepts in the language of nets and filter, rather than sequences and neighborhoods.

\ref{nf as set-theoretic objects} is a preparatory section, where basic concepts and results, such as maximal filter (net), its characterization, and derivation, etc., are introduced. It soon turns out that derivation is a convenient tool to investigate the properties of nets and filters at the same time. For instance, in \ref{maximal net} we see existence and characterization of maximal net immediately follow from the corresponding results on filter through derivation.

\ref{in topology} is our main part, which includes the characterization of various topological concepts, such as continuity, compactness, and Hausdorff separation axiom, in the language of nets and filters.
The section closes with an application to compact space, in which filter-based proofs are provided to well-known statements. The highlight is a simple proof of Tychonoff's Thereom.

More extensive treatment on the subject can be found in \cite{bour:tplgy}, \cite{nagata:tplgy}.
\end{document}