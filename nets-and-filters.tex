\documentclass[a4paper,12pt]{article}
% Math Packages
\usepackage{amsmath,amsfonts}
\usepackage{mathrsfs}
\usepackage{amssymb}
\usepackage{amsthm}

% ---------- Environment
\newtheorem{thm}{Theorem}[section]
\newtheorem*{thm*}{Theorem}

\newtheorem{prp}[thm]{Proposition}
\newtheorem*{prp*}{Proposition}

\newtheorem{cor}[thm]{Corollary}
\newtheorem*{cor*}{Corollary}

\newtheorem{lem}[thm]{Lemma}
\newtheorem*{lem*}{Lemma}

\newtheorem{dfn}[thm]{Definition}
\newtheorem*{dfn*}{Definition}

\theoremstyle{remark}
\newtheorem*{prf}{Proof}

\theoremstyle{definition}
\newtheorem*{rem*}{Remark}
\newtheorem{rem}[thm]{Remark}

\theoremstyle{definition}
\newtheorem{ex}[thm]{Example}
\newtheorem*{ex*}{Example}

\theoremstyle{definition}
\newtheorem{exe}[thm]{Exercise}
% --------- Environment

% --------- macros
\newcommand{\ip}[2]{\left<#1, #2 \right>}
\newcommand{\adj}[1]{#1^{\star}}
\newcommand{\interior}[1]{%
  {\kern0pt#1}^{\mathrm{o}}%
}
% --------- end macros

\begin{document}

\section*{Overview}
The author have made some effort to keep every proof simple and clear and detailed so that free of no annoying complexity, in order to make it easier even for a forgetful person to catch up the argument again.

\section{Quick Introduction to Filter}

\subsection{Overview}

\subsection{Basic Definition}
\begin{dfn} (Filter and maximal filter)

	A nonempty collection \( \mathscr{F} \) of subsets of a set \( X \) is called a filter of \( X \) if it satisfies
	\begin{itemize}
		\item[(1)] \( \emptyset \in \mathscr{F} \).
		\item[(2)] \( X \supset B \supset A \in \mathscr{F} \implies B \in \mathscr{F}\).
		\item[(3)] \( A, B \in \mathscr{F} \implies A \cap B \in \mathscr{F}\).
	\end{itemize}

	Given two filters \( \mathscr{F} \) and \( \mathscr{G} \) of the same set, \( \mathscr{G} \) is called finer than \( \mathscr{F} \), or \( \mathscr{F} \) is called coarser than \( \mathscr{G} \), if \( \mathscr{G} \supset \mathscr{F}\), and said to be strictly so if the inclusion is strict.
\end{dfn}

\begin{dfn} (Maximal, ultra-filter)
	A filter \( \mathscr{F} \) of \( X \) is called a maximal (or ultra) filter if it admits no strictly finer filter of \( X \).
\end{dfn}

\begin{rem}
	Every filter \( \mathscr{F} \) has the following finite intersection property; for every finitely many member \( F_1 ,\ldots, F_n \) of \( \mathscr{F} \) there holds \( \cap F_i \neq \emptyset \).
\end{rem}

\begin{dfn} (Convergence of filter)
	We say that filter \( \mathscr{F} \) converges to a point \( p \), and write \( \mathscr{F} \to p \), if every neighborhood \( N \) of \( p \) belongs to \( \mathscr{F} \).
\end{dfn}

\begin{rem}\label{rem filter convergence}
	Note (verify!) that \( \mathscr{F} \to p \) if and only if for every neighborhood \( N \) of \( p \) there is \( F \in \mathscr{F} \) such that \( F \subset N \).
\end{rem}

\begin{dfn} (Cluster point of filter)
	A point \( p \) is called a cluster point of a filter \( \mathscr{F} \) if \( p \) lies in the closure of every member of \( \mathscr{F} \).
\end{dfn}

As is often the case with other concept in mathematics, a filter can be generated by its smaller subset.
\begin{dfn} (Filter basis)
	A nonempty collection \( \mathscr{B} \) of subset of a set \( X \) is called a filter basis if it has the following property;
	\begin{itemize}
		\item[(1)] \( \emptyset \notin \mathscr{B} \).
		\item[(2)] For \( B_1, B_2 \in \mathscr{B} \), there is \( B_3 \in \mathscr{B} \) such that \( B_3 \subset B_1 \cap B_2 \).
	\end{itemize}

	A filter basis \( \mathscr{B} \) is called a basis of a filter \( \mathscr{F}  \) if for every \( F \in \mathscr{F} \) there is \( B \in \mathscr{B} \) such that \( B \subset F \).
\end{dfn}



\begin{rem}
	Given a filter basis \( \mathscr{B} \), the filter \( \mathscr{F} \) defined by
	\begin{equation*}
		\mathscr{F} = \{F \mid F \supset B,\,\exists B \in \mathscr{B}\}
	\end{equation*}
	contains \( \mathscr{B} \) as it subcollection. It is easy to see that \( \mathscr{B} \) is a basis of \( \mathscr{F} \). Whence \( \mathscr{B} \) is said to generate \( \mathscr{F} \).
\end{rem}

The typical set theoretic argument establishes the following minimality property.
\begin{prp}\label{coarsest generation}
	The filter generated by a filter basis is the coarsest filter containing the basis.
\end{prp}

\begin{ex} (Filter generated by a set)
	\( \mathscr{B}:=\{A\} \) with \( A \neq \emptyset \) is the simplest filter basis. The filter generated by \( \mathscr{B} \) is called the filter generated by \( A \), and denoted by \( \left< A \right> \).
\end{ex}

\begin{dfn} (Convergence and cluster point of filter basis)
	We say that a filter basis \( \mathscr{B} \) converges to a point \( p \), and write \( \mathscr{B} \to p \), if for every neighborhood \( N \) of \( p \) there is \( B \in \mathscr{B} \) such that \( B \subset N \).

	A point \( p \) is called a cluster point of a filter basis \( \mathscr{B} \) if \( p \in \overline{B} \) for every \( B \in \mathscr{B} \).
\end{dfn}

\begin{rem}
	Here we adopt, as definition of convergence of filter basis, the necessary and sufficient condition of convergence of filter. See remark\ref{rem filter convergence}.
\end{rem}

\begin{rem}
	Clearly, \( p \) is a cluster point of a filter basis \( \mathscr{B} \) if and only if \( \mathscr{B} \) has trace on every neighborhood of \( p \).
\end{rem}

\begin{ex}\label{example local basis} (basis of neighborhoods as a filter)
	Suppose \( \mathscr{N}(p) \) is a basis of neighborhoods of a point \( p \) of a topological space.
	\( \{p\} \cup \mathscr{N}(p) \) is a filter basis, but not necessarily a filter. On the other hand, \( \mathscr{N}(p) \) is a filter. Both of them converge to \( p \).
\end{ex}

In general, \( \mathscr{F}\to p \) implies \( \mathscr{G}\to p \) for a filter \( \mathscr{G} \supset \mathscr{F} \), and not conversely. The following proposition gives a special example in which the converse holds.
\begin{prp}
	Suppose a filter basis \( \mathscr{B} \) generates a filter \( \mathscr{F} \). Then \( \mathscr{B}\to p \) if and only if \( \mathscr{F}\to p \).
\end{prp}
\begin{prf}
	Suppose \( \mathscr{F}\to p \). By definition, for every neighborhood \( N \) of \( p \) there is \( F \in \mathscr{F} \) such that \( F \subset N \). For this \( F \) there is \( B \in \mathscr{B} \) such that \( B \subset N \) since \( \mathscr{B} \) generates \( \mathscr{F} \). Thus, \( \mathscr{B}\to p \). The converse is trivial.
	\qed\end{prf}

\begin{dfn} (Trace of filter basis)
	A filter basis is said to have trace on a set \( A \) if \( A \) intersects every member of the filter basis.
\end{dfn}

\begin{dfn} (Compatibility)
	Two filter bases \( \mathscr{A} \) and \( \mathscr{B} \) are said to be compatible if \( A \cap B \neq \emptyset \) for every \( A \in \mathscr{A} \) and \( B \in \mathscr{B} \).
\end{dfn}

\begin{prp}\label{compatible} (Characterization of compatibility)
	Two filter bases \( \mathscr{F} \) and \( \mathscr{G} \) are compatible if and only if they admits a common finer filter.
\end{prp}
\begin{prf}
	If they are compatible, then
	\begin{equation*}
		\{F \cap G \mid F \in \mathscr{F},\,G \in \mathscr{G}\}
	\end{equation*}
	is a common finer filter.

	Conversely, suppose a common finer filter exists. Then every \( F \in \mathscr{F} \) and \( G \in \mathscr{G} \) is a member of the filter, and thus \( F \cap G \neq \emptyset \) necessarily follows from definition of filter.
	\qed\end{prf}

\begin{rem}\label{rem trace}
	If a filter \( \mathscr{F} \) has trace on \( A \), then we can construct a finer filter since the filter \( \left< A \right> \) generated by \( A \) is compatible with \( \mathscr{F} \). This procedure yields a strictly finer filter if and only if \( A \notin \mathscr{F}\). This fact is frequently exploited.
\end{rem}

% An important example of compatible classes is the convergent filter bases.
% \begin{prp}
% 	If each of two filter basis \( \mathscr{A} \) and \( \mathscr{B} \) converges to the same point, then they are compatible.
% \end{prp}
% \begin{prf}
% 	Suppose they converge to a point \( p \), and suppose they generate \( \mathscr{F} \) and \( \mathscr{G} \).
% 	It is easy to see that the generated filters also converge to \( p \).

% 	Since the relation \( F \cap G = \emptyset \) for some \( F \in \mathscr{F} \) and \( G \in \mathscr{G}\) means one of these filters cannot converge to \( p \), they must be compatible. Thus, the given filter bases are also compatible by definition of compatibility.
% 	\qed\end{prf}

\subsection{Maximality}

\begin{prp}\label{existence maximal filter} (Existence of maximal filter)
	If a collection has the finite intersection property, then there exists a maximal filter containing the collection.
\end{prp}
\begin{prf}
	Let \( \mathscr{C} \) be the set of a collection \( \mathscr{F} \) of sets with the finite intersection property and \( \mathscr{F} \supset \mathscr{A} \). Introduce an order on \( \mathscr{C} \) by the rule \( \mathscr{F}_1 \succeq \mathscr{F}_2 \iff \mathscr{F}_1 \supset \mathscr{F}_2\).
	It is easy to see that \( \mathscr{C} \) is partially ordered, and that every totally ordered subset admits a supremum. We then have a maximal element \( \mathscr{F} \) of \( \mathscr{C} \) by Zorn's Lemma.
	It turns out that \( \mathscr{F} \) so derived is a filter.

	Definition (1) of filter is obvious.
	For \( B \supset A \supset \mathscr{F} \), we see that \( \{B\}\cap \mathscr{F} \in \mathscr{C} \) and hence \( B \in \mathscr{F} \) by maximality.
	For \( A, B \in \mathscr{F} \), we similarly have \( \{A \cap B\} \cup \mathscr{F} \in \mathscr{C} \) by the finite intersection property, and thus \( A \cap B \in \mathscr{F} \) again by maximality.
	\qed\end{prf}

% \begin{rem}
% 	The proposition just stated provides easy access to an underlying maximal filter. For convenience, we sometimes prefer to working with a maximal filter instead of a non-maximal one, provided of course that the property of interact is invariant under this switch.
% \end{rem}

\begin{thm}\label{characterize maximal filter} (Characterization of maximal filter)
	Let \( \mathscr{F} \) be a filter in a set \( X \). The following conditions are equivalent.
	\begin{itemize}
		\item[(1)] \( \mathscr{F} \) is maximal.
		\item[(2)] Every set intersecting every member of \( \mathscr{F} \) belongs to \( \mathscr{F} \).
		\item[(3)] For any subset \( A \) of \( X \), either \( A \) or \( X \setminus A \) belongs to \( \mathscr{F} \).
	\end{itemize}
\end{thm}
\begin{prf}
	(1) \( \implies \) (2):
	Let \( A \) be such intersecting subset and let \( \mathscr{F} \) be maximal. Let \( \mathscr{C} \) be the collection of sets consisting of the sets \( C \) such that \( C \supset A \cap B \) for some \( B \in \mathscr{F} \). It is easy to check that \( \mathscr{C} \) is a filter, and that there holds \( \mathscr{F} \subset \mathscr{C} \) and \( A \in \mathscr{C} \). It then follows from maximality that \( \mathscr{F} = \mathscr{C} \), and hence \( A \in \mathscr{F} \).

	(2) \( \implies \) (3):
	Suppose (3) fails, that is, suppose \( A \notin \mathscr{F} \) and \( X \setminus A \in \mathscr{F} \) for some \( A \subset X\). It necessarily follows that \( F \nsubseteq A \) and \( F \nsubseteq X \setminus A \) for every \( F \in \mathscr{F} \). Then \( A \cap F \neq \emptyset \) and \( \left( X \setminus A \right) \cap F \neq \emptyset \) for every \( F \in \mathscr{F} \). Thus, (2) fails.

	(3) \( \implies \) (1):
	Suppose \( A \in \mathscr{G} \supset \mathscr{F} \), where \( \mathscr{G} \) is a filter. It necessarily follows from the definition of filter that \( X \setminus A \notin \mathscr{G} \), and hence \( X \setminus A \notin \mathscr{F} \). Thus, \( A \in \mathscr{F} \) by Assumption.
	\qed\end{prf}

\begin{prp} (Representation of filter)
	Every filter is the intersection of all finer maximal filter.
\end{prp}
\begin{prf}
	Suppose a filter \( \mathscr{F} \) is given. It is obvious that the intersection is finer than \( \mathscr{F} \). Conversely, suppose \( A \notin \mathscr{F}\). Then \( \mathscr{F} \) has trance on \( X \setminus A \) since \( A \) contains no members of \( \mathscr{F} \), from which it follows that \( \mathscr{F} \) admits a finer filter that contains \( X \setminus A \). Thus, every maximal filter finer than \( \mathscr{F} \) necessarily contains \( X \setminus A \).
	\qed\end{prf}

\subsection{Convergence}

\begin{lem}\label{Relation of convergence and cluster point} (Relation of convergence and cluster point)
	Let \( \mathscr{F} \) be a filter and \( p \) be a point.
	If \( \mathscr{F}\to p \), then \( p \) is a cluster point of \( \mathscr{F} \).
	Moreover, the converse is also true if \( \mathscr{F} \) is a maximal filter.
	Thus, if \( \mathscr{F}^{\star} \) is a maximal filter, \( p \) is a cluster point of \( \mathscr{F}^{\star} \) precisely when \( \mathscr{F}^{\star} \to p\).
\end{lem}
\begin{prf}
	Suppose \( \mathscr{F} \to p \). Let \( A \in \mathscr{F} \).
	Since every neighborhood \( N \) of \( p \) belongs to \( \mathscr{F} \), we have, by definition, \( A \cap N \neq \emptyset \), and hence \( p \in \overline{A} \). Thus, \( p \) is a cluster point of \( A \).

	Conversely, suppose \( p \) is a cluster point of a maximal filter \( \mathscr{F} \). Then, by definition of cluster point, every neighborhood \( N \) of \( p \) intersects every member of \( \mathscr{F} \). By Theorem\ref{characterize maximal filter}, we have \( N \in \mathscr{F} \), and therefore \( \mathscr{F} \to p \).
	\qed\end{prf}

% Now we are ready to state our first characterization result on convergence of filter. The conditions listed below are self-contained in the sense that they are all described solely in a language of filter. Later in the subsequent sections, we will provide another characterization of convergence from totally different point of view.
% \begin{prp} (Characterization of convergence of filter)
% 	Suppose \( \mathscr{F} \) is a filter, and let \( p \) be a point.
% 	The following conditions are all equivalent.
% 	\begin{itemize}
% 		\item[(1)] \( \mathscr{B}\to p \), where \( \mathscr{B} \) is a filter basis of \( \mathscr{F} \).
% 		\item[(2)] \( \mathscr{F}\to p \).
% 		\item[(3)] \( p \) is a cluster point of a maximal filter \( \mathscr{F}^{\star} \) that contains \( \mathscr{F} \).
% 		\item[(4)]  \( p \) is a cluster point of every filter \( \mathscr{G} \) that contains \( \mathscr{F} \).
% 	\end{itemize}
% \end{prp}
% \begin{prf}
% 	(1) \( \iff \) (2):
% 	For every neighborhood \( N \) of \( p \) there is \( B \in \mathscr{B} \subset \mathscr{F}\) such that \( B \subset N \), proving \( \mathscr{F} \to p \). Conversely, if \( \mathscr{F}\to p \), then for every neighborhood \( N \) of \( p \) there is \( F \in \mathscr{F} \) such that \( F \subset N \). For this \( F \) there is \( B \in \mathscr{B} \) such that \( B \subset N \) since \( \mathscr{B} \) generates \( \mathscr{F} \). Thus, \( \mathscr{B}\to p \).

% 	(2) \( \implies \) (3): If \( p \) is not a cluster point of a maximal filter \( \mathscr{F}^{\star} \), Lemma implies that \( \mathscr{F}^{\star} \) fails to converge to \( p \), and so does \( \mathscr{F} \).

% 	(3) \( \implies \) (4):
% 	Let \( \mathscr{F}^{\star} \) be a maximal filter that contains \( \mathscr{G} \). Assumption gives \( p \in \overline{F} \) for every \( F \in \mathscr{F}^{\star} \), in particular, for every \( F \in \mathscr{G} \). Thus, \( p \) is a cluster point of \( \mathscr{G} \).
% 	\qed\end{prf}

\section{Derivation}

\subsection{Overview}

Here is undoubtedly the most exciting part of this document, and indeed, of the theory of nets and filters.
In this section we see that every filter and net has its (essentially) unique twin. Each of individual filter and net turns into the corresponding twin by the conversion called derivation, and then comes back to the original one by another derivation. In other words, derivation is a kind of isomorphism between the space of filters and that of nets. In fact, as the readers reasonably expect, important properties, such as convergence and maximality and etc., are preserved under derivation. Nets and Filters are twins literally at the genetic level. So why not study them at the same time through derivation?

The main purpose of this section is to establish several crucial results that clarifies the genetic identity of nets and filters. As soon as the very basic definitions are introduced, the fundamental results are readily proved. We then deduce, as a simple application of them, a series of invariance results, such as that of convergence and maximality, followed by several fundamental results on maximal nets.

\subsection{Basic Definition}

% def net
\begin{dfn} (Net)
	Let \( \Lambda \) be a directed set and let \( X \) be a topological space.
	A net (on \( \Lambda \)) is a mapping \( x: \Lambda \ni \lambda \mapsto x_{\lambda}\in X \), and is denoted by \( (x_{\lambda})_{\lambda \in \Lambda} \) or \( x_{\lambda} \), or sometimes simply by \( x \) if there is no fear of confusion.
\end{dfn}

\begin{dfn} (Eventually and frequently)
	Let \( (x_{\lambda})_{\lambda \in \Lambda} \) be a net of a directed set \( \Lambda \) into a topological space \( X \). Suppose \( A \) is a subset of \( X \).
	\begin{itemize}
		\item \( x \) is said to be eventually in \( A \) or residual in \( A \) if there is \( \lambda_0 \in \Lambda \) such that \( x_{\lambda} \in A\) for all \( \lambda >\lambda_0 \).
		\item \( x \) is said to be frequently in \( A \) or cofinal in \( A \) if for every \( \lambda_0 \in \Lambda \) there is \( \lambda > \lambda_0\) such that \( x_{\lambda} \in A\).
		\item \( x_{\lambda} \) is called a maximal net or an ultra-net if \( x \) is eventually either in \( A \) or \( X \setminus A \) for every \( A \subset X \).
	\end{itemize}
\end{dfn}

\begin{dfn} (Convergence and cluster point of net)
	\begin{itemize}
		\item A net \( x_{\lambda} \) is said to converge to a point \( p \), and is denoted by \( x_{\lambda}\to p \), if \( x_{\lambda} \) is eventually in every neighborhood of \( p \).
		\item A point \( p \) is called a cluster point of a net \( x_{\lambda} \) if \( x \) is frequently in every neighborhood of \( p \).
	\end{itemize}
\end{dfn}

\begin{dfn} (Derived net)
	Let \( \mathscr{B} \) be a filter basis, and let \( \mathscr{F} = \{F_{\lambda} \mid \lambda \in \Lambda \} \) be the generated filter.
	A net \( x_{\lambda} \) derived from the filter basis \( \mathscr{B} \) is a mapping
	\begin{equation*}
		x: \Lambda \ni \lambda \mapsto x_{\lambda} \in F_{\lambda} \subset X,
	\end{equation*}
	where \( \Lambda \) is directed by the partial order \( > \) defined by \( \lambda_1> \lambda_2 \iff F_{\lambda_1} \subset F_{\lambda_2} \).
	We also say that \( \mathscr{F} \) (or \( \mathscr{B} \)) generates \( x \).
\end{dfn}

\begin{dfn} (Derived filter)
	The filter derived from a net \( x \) is the collection of the sets \( A \) in which \( x \) eventually lies, and is denoted by \( \mathscr{F}_x \). We also say that \( x \) generates \( \mathscr{F}_x \).
\end{dfn}

\begin{rem}
	Note that derived filter is uniquely specified as a filter once a net is designated.
	On the other hand, derived net no way specifies unique net except in trivial cases. In fact, derived net in general consists of a collection of projection nets of indexing set of a filter into a member of the filter.
\end{rem}

\subsection{Invariance under Derivation}

% \begin{thm}\label{invariance convergence} (Invariance of convergence under derivation)
% 	\begin{itemize}
% 		\item[(1)] A filter converges to a point if and only if every net derived from the filter converges to the point.
% 		\item[(2)] Conversely, a net converges to a point if and only if the derived filter converges to the point.
% 	\end{itemize}
% \end{thm}
% \begin{prf}
% 	(1):
% 	Suppose \( \mathscr{F} =\{F_{\lambda}\} \to p \). Let \( x_{\lambda} \) be a derived net. For every neighborhood \( N \) of \( p \) there is \( F_{\lambda_0} \in \mathscr{F} \) such that \( F_{\lambda_0} \subset N \). Since \( F_{\lambda} \subset F_{\lambda_0} \subset N \) for \( \lambda>\lambda_0 \), we have \( x_{\lambda} \in N \) for \( \lambda > \lambda_0 \).
% 	Thus, \( x \to p \).

% 	Conversely, suppose \( \mathscr{F} \) fails to converge to \( p \), that is, suppose there is neighborhood \( N \) of \( p \) such that no \( F_{\lambda} \) is contained in \( N \). Construct a net \( x_{\lambda} \) so that \( x_{\lambda} \in F_{\lambda} \setminus N \) for every \( \lambda \in \Lambda \).
% 	Then, \( x \) is a derived net not converging to \( p \).

% 	(2): Suppose \( \mathscr{F}_x \) is the derived filter from a net \( x \). Suppose \( x \to p \), that is, suppose \( x \) is eventually in \( N \) for every neighborhood \( N \) of \( p \). Then \( N \in \mathscr{F} \) by definition of \( \mathscr{F} \). Thus, \( \mathscr{F} \to p \).

% 	Conversely, suppose \( \mathscr{F}_x \to p \), i.e., every neighborhood \( N \) of \( p \) belongs to \( \mathscr{F} \). This means that \( x \) is eventually in \( N \). Thus, \( x \to p \).
% 	\qed\end{prf}

You can safely skip the next result. Actually, we will shortly establish several results that make it almost trivial, on which none on them depend. It only serves as an easy exercise about maximality.
\begin{exe}\label{invariance maximality} (Invariance of maximality under derivation)
	\begin{itemize}
		\item Every net derived from a maximal filter is maximal.
		\item Conversely, the filter derived from a maximal net is maximal.
	\end{itemize}
\end{exe}
\begin{prf}
	Suppose \( \mathscr{F} = \{F_{\lambda}\} \) is a maximal filter. For every \( A \subset X \), there holds either \( A \in \mathscr{F} \) or \( X \setminus A \in \mathscr{F} \). That is, every \( F_{\lambda} \) is of the form either \( A \) or \( X \setminus A \). Thus, every derived net is eventually in either \( A \) or \( X \setminus A\).

	Conversely, suppose a net \( x \) is maximal, that is, \( x \) is eventually in either \( A \) or \( X \setminus A \) for every \( A \subset X \). Then, by definition, the derived filter \( \mathscr{F} \) consists of \( A \) or \( X \setminus A \). Thus, \( \mathscr{F} \) is maximal.
	\qed\end{prf}

\begin{dfn} (Subnet, \( \succeq \))
	Let \( x_{\lambda} \) be a net in a set \( X \). We say that a net \( \left( s_d \right)_{d \in D} \) is a subnet of \( x_{\lambda} \), and write \( s \succeq x \), if for every \( \lambda \in \Lambda \) there is \( d \in D \) such that
	\begin{equation*}
		s(D_d) \subset x(\Lambda_{\lambda}),
	\end{equation*}
	where \( D_d:=\{ e \in D \mid e >d \} \) and \( \Lambda_{\lambda} \) is defined similarly.

	If \( s \) is a subnet of a net \( x \) but not the converse, then \( s \) is called a proper subnet of \( x \).
\end{dfn}

\begin{dfn} (Relation of subnet: \( \sim \))
	Two nets \( x \) and \( y \) are called equivalent, and are denoted by \( x \sim y \), if they are subnets of each other, that is, if \( x \succeq y \) and \( y \succeq x \).
\end{dfn}

\begin{rem}
	It is obvious that the relation \( \succeq \) fulfills the axiom of partial order, and that the equivalence \( \sim \) of subnet fulfills the axiom of equivalence relation.
\end{rem}

The next Lemma immediately follows from definition.
\begin{lem}\label{classification by eventual} (relation induced by eventual behavior)
	\begin{itemize}
		\item A net \( g \) is a subnet of a net \( f \) if and only if \( g \) is eventually in \( A \) whenever \( f \) is eventually in \( A \).
		\item Two nets are equivalent if they share the same derived filter, that is, they eventually lie in the same set.
	\end{itemize}
\end{lem}

\begin{lem}\label{reflexive property} (Idempotent property of derivation)
	\begin{itemize}
		\item[(1)] If \( x \) is a net derived from a filter \( \mathscr{F} \), then \( \mathscr{F} = \mathscr{F}_x \).
		\item[(2)] If \( \mathscr{F}_x \) is the filter derived from a net \( x \),
		      then every net \( f \) derived from \( \mathscr{F}_x \) is equivalent to \( x \).
	\end{itemize}
\end{lem}
\begin{prf}
	(1): Write \( x = (x_{\lambda}) \). Observe \( A \in \mathscr{F}_x \) if and only if \( x \) is eventually in \( A \).
	This is the case if and only if for every \( \lambda_0 \) we have \( x_{\lambda} \in A \) for all \( \lambda > \lambda_0 \).
	This holds if and only if there exists a subset \( F_{\lambda}\in \mathscr{F} \) such that \( x_{\lambda} \in F_{\lambda} \subset A \), from which it follows that \( A \in \mathscr{F} \), and conversely (consider contraposition).

	(2): Note that the filter \( \mathscr{F} \) derived from \( f \) coincides with \( \mathscr{F}_x \) by (1). Thus \( x \) and \( f \) are equivalent by Lemma\ref{classification by eventual}.
	\qed\end{prf}

The following theorem and its corollary are the highlight of this section. The proof is almost obvious since it is just a rephrase of the above Lemma.
\begin{thm}\label{fundamental theorem of derivation} (Fundamental theorem of derivation)
	\begin{itemize}
		\item[(1)] For every filter there exists a net that generates the filter. Moreover, the net is unique up to the subnet-equivalence modulo.
		\item[(2)] For every net there exists unique filter that generates a net equivalent to the given net.
	\end{itemize}
\end{thm}
\begin{prf}
	Lemma\ref{reflexive property} has establishes the existence part. It remains to show the uniqueness.
	To prove the first claim, let \( \mathscr{F} \) be a filter, and suppose two nets \( x \) and \( y \) generates the identical filter \( \mathscr{F} \). But this is just rephrase of \( x \sim y \) by definition.
	Similarly, the second claim immediately follows from Lemma\ref{reflexive property}.
	\qed\end{prf}

\begin{rem} (Implication of fundamental theorem of derivation)
	Theorem\ref{fundamental theorem of derivation} say that derivation works like an idempotent bijection between the set of filters and that of equivalence classes of nets, as we have suggested at the beginning of the section. It is thus natural to introduce a notation that represents the paired relationship, as follows.
\end{rem}

\begin{dfn} (Equivalence of net and filter: \( \simeq \))
	We say that a net \( x \) and a filter \( \mathscr{F} \) are equivalent, and write \( x \simeq \mathscr{F} \), if \( \mathscr{F}_x = \mathscr{F} \).
\end{dfn}

\begin{cor} (Invariance of order under derivation)\label{invariance order}
	Suppose filters \( \mathscr{F} \), \( \mathscr{G} \) and nets \( f \), \( g \) have the following equivalence relations:
	\[
		\mathscr{F} \simeq f,\quad \mathscr{G} \simeq g.
	\]
	Then we have the following equivalence:
	\[
		\mathscr{G} \supset \mathscr{F} \iff g \succeq f.
	\]
\end{cor}
\begin{prf}
	By Theorem\ref{fundamental theorem of derivation}, we may assume that \( \mathscr{F} \) and \( \mathscr{G} \) are derived from \( f \) and \( g \), respectively. \( g \) is a subnet of \( f \) if and only if, by Lemma\ref{classification by eventual}, \( g \) is eventually in \( A \) whenever \( f \) is eventually in \( A \).
	It then follows from Theorem\ref{fundamental theorem of derivation} and definition of derived filter that \( \mathscr{G} = \mathscr{F}_g \supset \mathscr{F}_f = \mathscr{F} \), and conversely.
	\qed\end{prf}

\subsection{Maximality of Nets}
As an easy application of the preceding results, here we establish several results regarding the maximality of nets.

\begin{rem}
	It's the best time to look back Exercise\ref{invariance maximality} you might have skipped, which claims that maximality is invariant under derivation. Now it is an immediate consequence of Corollary\ref{invariance order} and Proposition\ref{characterize maximal net}.
\end{rem}

\begin{thm}\label{characterize maximal net} (Characterization of maximal net)
	Let \( x \) be a net. The following conditions are equivalent.
	\begin{itemize}
		\item[(1)] \( x \) is eventually either in \( A \) or \( X \setminus A \) for every \( A \subset X \).
		\item[(2)] \( x \) admits no proper subsets.
		\item[(3)] The derived filter \( \mathscr{F}_x \) is maximal.
	\end{itemize}
\end{thm}
\begin{prf}
	(1) \( \iff  \) (3) by Exercise\ref{invariance maximality}.

	(2) \( \iff  \) (3): By Corollary\ref{invariance order}, \( x \) admits no proper subnets if and only if \( x \) is equivalent to a maximal filter.
	Thus \( \mathscr{F}_x \) is maximal, and conversely.
	\qed\end{prf}

The existence of a maximal subnet has also become an easy exercise.
\begin{cor} (Existence of maximal subnet)
	Every net admits a subnet which has no proper subnet (or equivalently, is maximal).
\end{cor}
\begin{prf}
	For every net \( x \), the derived filter \( \mathscr{F}_x \) admits a maximal filter \( \mathscr{F}^{\star} \) by Proposition\ref{existence maximal filter}, which in turn admits a corresponding maximal net \( y \) by Exercise\ref{invariance maximality}. Whence we have
	\[
		x \simeq \mathscr{F}_x \subset \mathscr{F}^{\star} \simeq y,
	\]
	and hence \( x \succeq y \) by Corollary\ref{invariance order}. This means \( y \) is a subnet of \( x \).
	\qed\end{prf}

\subsection{Convergence}

\begin{thm}\label{invariance convergence} (Invariance of convergence under derivation)
	Let \( \mathscr{F} \) be a filter and let \( x \) be a net. Suppose \( p \) is a point. If \( \mathscr{F} \simeq x \), then we have the following equivalence:
	\[
		\mathscr{F} \to p \iff x \to p.
	\]
\end{thm}
\begin{prf}
	By Theorem\ref{fundamental theorem of derivation}, we may assume that \( \mathscr{F} \) is derived from \( x \). For every neighborhood \( N \) of \( p \), we see that \( x \to p \) implies \( x \) is eventually in \( N \), and hence \( N \in \mathscr{F} \), and thus \( \mathscr{F}\to p \), and conversely.
	\qed\end{prf}

\begin{exe} (Convergence of subnet)
	If a net converges to a point, so is every subnet of it.
\end{exe}

An analogous argument to the sequence case shows the following Proposition.
\begin{exe}\label{fe relation} (Frequent-eventual relation)
	A net \( x \) is frequently in a set \( A \) if and only if \( x \) admits a subnet that is eventually in \( A \).
\end{exe}

\section{Topology with Nets and Filters}

\subsection{Overview}

\subsection{Transformation of Filter}

\begin{dfn} (Image of filter)
	Let \( f:X \to Y \) be a map.
	The image of a filter \( \mathscr{F} \) of \( X \) under \( f \) is the filter generated by the filter basis
	\begin{equation*}
		I_f(\mathscr{F}):=\{f(F) \mid F \in \mathscr{F}\},
	\end{equation*}
	and is denoted by \( f(\mathscr{F}) \). The image of a filter basis \( \mathscr{B} \) is defined to be \( I_f(\mathscr{B}) \) itself, and is also denoted by \( f(\mathscr{B}) \).
\end{dfn}

\begin{exe}
	Let \( f:X \to Y \) be a map, and let \( \mathscr{F} \) be a filter of \( X \). If \( f \) is surjective, then \( f(\mathscr{F}) = I_f(\mathscr{F}) = \{G \subset Y \mid f^{-1}(G) \in \mathscr{F}\} \).
\end{exe}

\begin{exe}
	Let \( f:X \to Y \) be a map, and let \( \mathscr{A} \) and \( \mathscr{B} \) be filter bases of \( X \). Then \( \mathscr{A} \subset \mathscr{B} \) implies \( f(\mathscr{A}) \subset f(\mathscr{B}) \), and conversely if \( f \) is injective. (Use the fact that \( f^{-1}(f(A)) =A \) if \( f \) is injective.)
\end{exe}

\begin{exe}\label{inverse image of filter} (Inverse image of filter basis)
	Let \( f:X \to Y \) be a map, and let \( \mathscr{B} \) be a filter basis of \( Y \). Then \( f^{-1}(\mathscr{B}):=\{f^{-1}(B) \mid B \in \mathscr{B}\} \) is a filter basis of \( X \) if and only if \( \mathscr{B} \) has trace on \( f(X) \).

	When one of these conditions are fulfilled, \( f^{-1}(\mathscr{B}) \) is called the inverse image of \( \mathscr{B} \) under \( f \). We  use the same notation for the filter generated by the filter basis \( f^{-1}(\mathscr{B}) \).
\end{exe}

\begin{exe}\label{double mapping}
	Let \( f:X \to Y \) be a map. Suppose \( \mathscr{B}_X \) and \( \mathscr{B}_Y \) are filter bases of \( X \) and \( Y \), respectively.
	Then the following relation holds, provided that each of the inverse image is properly defined as a filter basis.
	\begin{itemize}
		\item[(1)] \( f(f^{-1}(\mathscr{B}_Y)) \supset \mathscr{B}_Y \).
		\item[(2)] \( f^{-1}(f(\mathscr{B}_X)) \subset  \mathscr{B}_X \), with equality if \( f \) is injective.
	\end{itemize}
\end{exe}

\begin{exe}\label{inverse image preserve order}
	Let \( f:X \to Y \) be a surjective map, and let \( \mathscr{A} \) and \( \mathscr{B} \) be filter bases of \( Y \). Then \( \mathscr{A} \subset \mathscr{B} \) if and only if \( f^{-1}(\mathscr{A}) \subset f^{-1}(\mathscr{B}) \), where the inverse images are considered as filter bases. (Use the fact that \( f(f^{-1}(A)) =A \) if \( f \) is surjective.)
\end{exe}

\begin{prp}\label{invariance under mapping} (Invariance of equivalence under mapping)
	Let \( f:X \to Y \) be a map, and let \( x \) be a net and \( \mathscr{F} \) a filter. Then
	\[
		x \simeq \mathscr{F} \implies f(x)\simeq f(\mathscr{F}).
	\]
\end{prp}
\begin{prf}
	By Theorem\ref{fundamental theorem of derivation}, we may assume that the filter is derived by the net, that is, \( \mathscr{F} = \mathscr{F}_x \).
	It suffices to show that
	\begin{equation*}
		f^{-1}(A)\in \mathscr{F}_x \iff A \in f(\mathscr{F}_x).
	\end{equation*}
	(\( \implies \)) is obvious. The converse is also obvious for a filter basis \( \mathscr{B}=I_f(\mathscr{F}_x) \) that generates \( f(\mathscr{F}_x) \). It therefore follows that for every \( A \in f(\mathscr{F}_x)\) we can choose \( B \in \mathscr{B} \) such that \( B \subset A \) and \( f^{-1}(B) \in \mathscr{F}_x \), which gives \( f^{-1}(A) \in \mathscr{F}_x \).
	\qed\end{prf}

\begin{prp}\label{invariace of maximality under mapping} (Invariance of maximality under mapping)
	Let \( f:X \to Y \) be a surjective map. If a filter \( \mathscr{F} \) of \( X \) is maximal, so is \( f(\mathscr{F}) \).
\end{prp}
\begin{prf}
	Let \( G \subset Y \). By maximality, there holds either \( f^{-1}(G)\in \mathscr{F} \) or not. If this is the case, it follows that \( f^{-1}(G) = F \) for some \( F \in \mathscr{F} \), and hence \( G \supset f(f^{-1}(G))=f(F) \), implying \( G \in f(\mathscr{F}) \). If not, we have \( X \setminus f^{-1}(G) = f^{-1}(X \setminus G) \in \mathscr{F} \). Thus, \( X \setminus G \in f(\mathscr{F}) \).
	\qed\end{prf}

\begin{ex} (Counterexample to the converse)
	Here we see how the converse fails. Let \( \mathscr{N} \) be the set of open intervals over 0 in real line:
	\[
		\mathscr{N} := \{(a,b) \subset \mathbb{R} \mid a<0<b\}.
	\]
	Observe that, by Proposition\ref{characterize maximal filter}, \( \mathscr{N} \) is not maximal while \( \mathscr{N}\cup \{0\} \) is.
	Define \( f:\mathbb{R}\to \mathbb{R}_+ \) via
	\[
		f(x):=\max \{|x|-1,0\}.
	\]
	Then we have
	\[
		f(\mathscr{N}) =\{[0,c) \subset \mathbb{R}_+ \mid c>0\} \cup \{0\} = f(\mathscr{N}\cup \{0\}).
	\]
	Proposition\ref{invariace of maximality under mapping} shows that \( f(\mathscr{N}) = f(\mathscr{N}\cup \{0\}) \) is maximal.
\end{ex}

% \begin{prp} (invariance of maximality under inverse image)
% 	Let \( f:X \to Y \) be a surjective map. Suppose \( \mathscr{F} \) be a filter of \( Y \).  If the filter \( f^{-1}(\mathscr{F}) \) is maximal, so is \( \mathscr{F} \).
% \end{prp}
% \begin{prf}
% 	Suppose \( f^{-1}(\mathscr{F}) \) is maximal. If \( \mathscr{F} \subset \mathscr{G} \) for some filter \( \mathscr{G} \), then Exercise\ref{inverse image preserve order} gives \( f^{-1}(\mathscr{F}) \subset f^{-1}(\mathscr{G}) \). But assumption tells us that this inclusion holds with equality. Thus, Exercise\ref{inverse image preserve order} yields \( \mathscr{F} =\mathscr{G} \).
% 	\qed\end{prf}

\begin{dfn} (Induced filter)
	Suppose \( X \) is a set, and suppose \( (X_i)_{i \in I} \) is an indexed set. Let \( \mathscr{F}_i \) be a filter on \( X_i \) and \( \pi_i:X \to X_i \) a map such that \( \pi_i ^{-1}(\mathscr{F}_i) \) is a filter basis of \( X \) for every \( i \in I \). Then the filter \( \mathscr{F} \) induced on \( X \) by \( (\mathscr{F}_i ,\pi_i)_{i \in I} \) is the coarsest one satisfying
	\begin{equation*}
		\pi_i(\mathscr{F}) = \mathscr{F}_i
	\end{equation*}
	for every \( i \in I \).
\end{dfn}

\begin{ex} (Product filter)
	In the above definition, take \( \pi_i :X \ni x = (x_i) \mapsto x_i \in X_i \) as the natural projection. The resulting filter is called the product filter of \( \mathscr{F}_i \), and is denoted by \( \prod \mathscr{F}_i \). Note that each \( \pi_i \) is surjective, and therefore the inverse image of filter is automatically well-defined.
\end{ex}

\begin{ex} (Restriction of filter onto trace)
	Suppose a filter \( \mathscr{F} \) of \( X \) has trace on \( A \). Then the pair of \( \mathscr{F} \) and an inclusion map \( i_A:A \to X \) fulfills the condition of Exercise\ref{inverse image of filter}. The resulting induced filter of \( A \) given by \(i_A^{-1}(\mathscr{F}) =\{F \cap A \mid F \in \mathscr{F}\} \) is called the restriction of \( \mathscr{F} \) onto \( A \).
\end{ex}

\begin{prp}\label{characterize induced filter} (Characterization of induced filter)
	Let \( \mathscr{F} \) be a filter of \( X \) induced by the pair \( (\mathscr{F}_i ,\pi_i)_{i \in I} \) as in the above definition. Then \( \mathscr{F} \) is generated by a filter basis of \( X \) consisting of all subsets of the form \( \cap_{i \in J} \pi_i ^{-1}(F_i)\), where \( J \) is a finite subset of \( I \) and \( F_i \in \mathscr{F}_i \).
\end{prp}
\begin{prf}
	Let \( \mathscr{B} \) be the filter basis indicated in the claim, and let \( \mathscr{G} \) be the filter generated by \( \mathscr{B} \). It suffice to show that \( \mathscr{F}=\mathscr{G} \).

	Suppose \( A \in \mathscr{F}\). It follows from definition that, for each \( i \), there holds \( \pi_i(A) = F_i \) for some \( F_i \in \mathscr{F}_i \), and hence \( A \supset \pi_i ^{-1}(F_i) \), in which the last term is a member of the basis of \( \mathscr{G} \). Thus, \( \mathscr{F} \subset \mathscr{G} \). To prove the opposite inclusion, we claim \( \pi_i(\mathscr{B})\subset \mathscr{F}_i \) for every index \( i \), from which we deduce that \( \pi_i(\mathscr{G})\subset \mathscr{F}_i \) for every index \( i \), and thus \( \mathscr{G} \subset \mathscr{F} \) by minimality of \( \mathscr{G} \). Suppose \( A \in \mathscr{B}\), that is, suppose \( A \) is of the form
	\[
		A = \bigcap_{j \in J} \pi_j^{-1}(F_j),
	\]
	where \( J \) is a finite subset of \( I \), and \( F_j \in \mathscr{F}_j \). Then we have
	\[
		\pi_j(A) \subset \bigcap_{j \in J}F_j \subset F_j \in \mathscr{F}_j
	\]
	for every \( j \in J \), from which our claim follows.
	\qed\end{prf}

\subsection{Characterization of Topological Properties}

\begin{prp}\label{characterize cluster point} (Characterization of cluster point)
	Suppose \( p \) is a point of a topological space \( X \). Let \( x \) be a net of \( X \), and let \( \mathscr{F} \) be the filter of \( X \) equivalent to \( x \). The following conditions are equivalent.
	\begin{itemize}
		\item[(1)] \( p \) is a cluster point of \( x \).
		\item[(2)] \( p \) is a cluster point of \( \mathscr{F} \).
		\item[(3)] \( x \) admits a subnet converging to \( p \).
		\item[(4)] There is a filter \( \mathscr{G} \) such that \( \mathscr{G} \supset \mathscr{F} \) and \( \mathscr{G} \to p \).
	\end{itemize}
\end{prp}
\begin{prf}
	(1) \( \iff \) (3): By Proposition\ref{fe relation}.

	(3) \( \iff \) (4): By invariance of order (Corollary\ref{invariance order})  and invariance of convergence (Theorem\ref{invariance convergence}).

	(2) \( \iff \) (4): \( p \) is a cluster point of \( \mathscr{F} \) precisely when \( \mathscr{F} \) has trace on every \( N \in \mathscr{N} \), where \( \mathscr{N} \) is a basis of neighborhood of \( p \). This is the case precisely when \( \mathscr{F} \) and \( \mathscr{N} \) are compatible, which is true precisely when (4) holds by Proposition\ref{compatible}.
	\qed\end{prf}

\begin{prp} (Characterization of neighborhood)
	Suppose \( p \) is a point of a topological space \( X \), and suppose \( U \) is a subset of \( X \). The following conditions are equivalent.
	\begin{itemize}
		\item[(1)] \( U \) is a neighborhood of \( p \).
		\item[(2)] \( U \cap \{x\} \neq \emptyset \) for every net \( x \) converging to \( p \).
		\item[(3)] Every filter converging to \( p \) contains \( U \) as its member.
	\end{itemize}
\end{prp}
\begin{prf}
	(1) \( \implies \)(2): Obvious.

	(2) \( \implies \)(3):
	Let \( \mathscr{F} \) be a filter with \( U \notin \mathscr{F} \to p \). Every net derived \( x \) from \( F \) converges to \( p \) but \( U \cap \{x\} = \emptyset \).

	(3) \( \implies \) (1): Let \( \mathscr{N}:=\mathscr{N}(p) \) be the set of all neighborhoods of \( p \). Clearly, \( \mathscr{N}\to p \), and thus \( U \in \mathscr{N} \) by assumption.
	\qed\end{prf}

\begin{prp}\label{characterize closure} (Characterization of closure)
	Suppose \( A \) is a subset of a topological space \( X \). Let \( p \) be a point in \( X \). The following conditions are equivalent.
	\begin{itemize}
		\item[(1)] \( p \in \overline{A} \).
		\item[(2)] There is a net of \( A \) converging to \( p \).
		\item[(3)] \( p \) is a cluster point of a filter of \( A \).
		\item[(4)] There is a filter basis of \( A \) converging to \( p \).
	\end{itemize}
\end{prp}
\begin{prf}
	Let \( \mathscr{N}(p) \) be the set of all neighborhoods of \( p \).
	(1) is true if and only if \( \mathscr{N}(p) \) has trace on \( A \). This is the case if and only if (4) holds. Other equivalences are easy to deduce.
	\qed\end{prf}

\begin{exe}\label{characterize closed} (Characterization of closed sets)
	For a set \( A \) in a topological space \( X \), the following conditions are equivalent.
	\begin{itemize}
		\item[(1)] \( A \) is closed.
		\item[(2)] If a filter \( \mathscr{F} \) of \( A \) converges to a point \( p \), then \( p \in A \).
	\end{itemize}
\end{exe}

\begin{exe}\label{continous mapping of cp} (Continuous mapping of cluster point)
	Suppose \( f:X \to Y \) is continuous, and suppose a filter \( \mathscr{F} \) has a cluster point \( p \). Then the filter \( f(\mathscr{F}) \) has \( f(p) \) as one of its cluster points.
\end{exe}

\begin{prp}\label{characterize continuity} (Characterization of continuity)
	Let \( f:X \to Y \) be a map. The following conditions are equivalent.
	\begin{itemize}
		\item[(1)] \( f \) is continuous at a point \( p \).
		\item[(2)] For any filter basis \( \mathscr{B} \) of \( X \), if \( \mathscr{B} \to p \), then \( f(\mathscr{B}) \to f(p) \).
		\item[(3)] For any filter \( \mathscr{F} \) of \( X \), if \( \mathscr{F} \to p \), then \( f(\mathscr{F})\to f(p) \).
		\item[(4)] For any net \( x \) of \( X \), if \( x \to p \), then \( f(x)\to f(p) \).
	\end{itemize}
\end{prp}

\begin{prf}
	(1) \( \implies  \) (2):
	Suppose \( N \) is a neighborhood of \( f(p) \). By assumption, there is a neighborhood \( U \) of \( p \) such that \( f(U) \subset N \). But by continuity, there is \( B \in \mathscr{B} \) such that \( B \subset U \), and thus \( f(B) \subset f(U) \subset N \).

	(2) \( \implies \) (3): Obvious.

	(3) \( \implies \) (4):
	Since \( \mathscr{F}_x \to p \) by assumption and Proposition\ref{invariance convergence}, it follows from (3) and Exercise\ref{invariance under mapping} that \( f(\mathscr{F}_x) \simeq f(x)\to p \).

	(4) \( \implies \) (1): Take net as an usual sequence.
	\qed\end{prf}

\begin{prp}\label{characterize compact} (Characterization of compactness)
	Let \( X \) be a topological space. The following conditions are equivalent.
	\begin{itemize}
		\item[(1)] \( X \) is compact.
		\item[(2)] Every closed collection of subsets of \( X \) with the finite intersection property has a nonempty intersection.
		\item[(3)] Every filter of \( X \) has a cluster point.
		\item[(4)] Every maximal filter is convergent.
	\end{itemize}
\end{prp}
\begin{prf}
	(1) \( \iff \) (2):
	Every elementary course on topology should contain this result, so the proof is left to the reader.

	(2) \( \implies \) (3): For every filter \( \mathscr{F} \), define
	\begin{equation*}
		\overline{\mathscr{F}} := \{\overline{F} \mid F \in \mathscr{F}\}.
	\end{equation*}
	\( \overline{\mathscr{F}} \) is obviously a closed filter with the finite intersection property. Then (2) implies that there is a point \( p \) such that \( p \in \overline{F}\) for all \( F \in \mathscr{F} \). Thus, \( p \) is a cluster point of \( \mathscr{F} \).

	(3) \( \implies \) (4):
	Apply Lemma\ref{Relation of convergence and cluster point}.

	(4) \( \implies \) (2): Let \( \mathscr{C} \) be the collection of closed sets with the finite intersection property. By Proposition\ref{existence maximal filter}, there is a finer maximal filter, which is convergent by assumption. Suppose \( p \) is a limit point of it. Then \( p \) is a cluster point of the filter basis \( \mathscr{C} \). Thus, \( p \in \overline{C} = C \) for every \( C \in \mathscr{C} \).
	\qed\end{prf}

\begin{exe}
	Show (4) \( \implies \) (1) directly in the Proposition.
\end{exe}
\begin{prf}
	Suppose (4) holds. We prove that \textit{every covering \( \mathscr{U} \) of \( X \) that admits no finite subcovering is not open.}
	Let \( \mathscr{U} \) be such a covering. Observe that
	\begin{equation*}
		\mathscr{B}:=\{X \setminus U \mid U \in \mathscr{U}\}
	\end{equation*}
	has the finite intersection property. We can therefore, by Proposition\ref{existence maximal filter}, construct a maximal filter \( \mathscr{F} \supset \mathscr{B} \). By (4), we have \( \mathscr{F}\to p \) for some \( p \in X \), which is a cluster point of \( \mathscr{F} \) by Proposition\ref{characterize cluster point}. Then, by definition of cluster point, we have \( p \in \overline{F} \) for every \( F \in \mathscr{F} \), in particular, \( p \in \overline{X \setminus U} \) for every \( U \in \mathscr{U} \).
	On the other hand, we have \( p \in U \), that is, \( p \in X \setminus U \) for some \( U \in \mathscr{U} \) since \( \mathscr{U} \) is a covering of the whole space. This means \( \overline{X \setminus U} \supsetneq X \setminus U \) for some \( U \in \mathscr{U} \). Thus, \( \mathscr{U} \) is not open.
	\qed\end{prf}

\begin{prp}\label{characterize Hausdorff} (Characterization of Hausdorff separation axiom)
	The following conditions about a topological space \( X \) are equivalent;
	\begin{itemize}
		\item[(1)] \( X \) is a Hausdorff space.
		\item[(2)] Every filter basis of \( X \) converges at most one point.
		\item[(3)] Every filter basis of \( X \) has at most one cluster point.
	\end{itemize}
\end{prp}
\begin{prf}
	Let \( \mathscr{N}(p) \) be a basis of neighborhoods of \( p \).
	(1) \( \implies  \) (2):
	Suppose a filter basis \( \mathscr{B} \) converges to a point \( p \). Pick a point \( q \) with \( q \neq p \). Then there exist \( U \in \mathscr{N}(p) \) and \( V \in \mathscr{N}(q)\) such that \( U \cap V = \emptyset \). Convergence of \( \mathscr{F} \) implies that there is \( B \in \mathscr{B} \) such that \( B \subset U \) and there is no \( B \in \mathscr{B}\) such that \( B \subset V \). Thus, \( \mathscr{B} \) converges to no point but \( p \).

	(2) \( \implies \)(3): Obvious by Proposition\ref{characterize cluster point}.

	(3) \( \implies \) (1):
	In a non-Hausdorff space, there are two distinct point \( p \) and \( q \) such that \( \mathscr{N}(p) \) and \( \mathscr{N}(q) \) are compatible. Then, \( p \) and \( q \) are cluster points.
	\qed\end{prf}

% \begin{thm}\label{characterize convergence init} (Characterization of convergence of initial topology)
% 	Suppose \( X \) is the topological space equipped with the initial topology induced by topological spaces \( X_{\lambda} \) and surjective maps \( \pi_{\lambda} \).
% 	Let \( \mathscr{F} \) be a filter of \( X \), and let \( p \) is a point.
% 	\begin{itemize}
% 		\item[(1)] \( \mathscr{F} \) converges to \( p \) if and only if each filter \( \mathscr{F}_{\lambda}:=\pi_{\lambda}(\mathscr{F}) \) converges to \( p_{\lambda}:= \pi_{\lambda}(p)\).
% 		\item[(2)] \( \mathscr{F} \) has a cluster point \( p \) if and only if each filter \( \mathscr{F}_{\lambda} \) has a cluster point \( p_{\lambda} \).
% 	\end{itemize}
% \end{thm}
% \begin{prf}
% 	(1): Necessity is obvious. Conversely, suppose \( F_{\lambda} \to p_{\lambda} \). Let \( N \) be a neighborhood of \( p \) in \( X \). By construction of initial topology, there is finitely many \( \lambda_1 ,\ldots, \lambda_k \) such that
% 	\begin{equation*}
% 		N \supset \bigcap_{i=1}^{k} \pi_{\lambda_1}^{-1}(N_{\lambda_i}),
% 	\end{equation*}
% 	where each \( N_{\lambda_i} \) is a neighborhood of \( p_{\lambda_i} \) of \( X_{\lambda_i} \). On the other hand, we have \( N_{\lambda_i} \in F_{\lambda_i} \) by assumption, that is, \( N_{\lambda_i} = \pi_{\lambda_i}(F_i)\) for some \( F_i \in \mathscr{F} \). From this it follows that \( F_i \subset \pi_{\lambda_i}^{-1} (N_{\lambda_i}) \), and hence \( \pi_{\lambda_i}^{-1} (N_{\lambda_i}) \in \mathscr{F} \), and therefore \( N \in \mathscr{F} \). Thus, \( \mathscr{F} \to p \).

% 	(2): The proof for (1) essentially includes that for the case of (2). So below is just kind of an exercise.

% 	Again, necessity is trivial. Conversely, suppose \( \mathscr{F} \) admits no cluster point. By Proposition\ref{characterize cluster point}, this is equivalent to saying that none of larger filter \( \mathscr{G} \) than \( \mathscr{F} \) converge to \( p \). By (1), none of \( \pi_{\lambda}(\mathscr{G}) \) converge to \( p_{\lambda_i} \). Since \( \pi_{\lambda}^{-1}(\mathscr{H}) \) is a filter of \( X \) with \( \pi_{\lambda}^{-1}(\mathscr{H}) \supset \mathscr{F} \) for every filter \( \mathscr{H} \) with \( \mathscr{H} \supset \mathscr{F}_{\lambda} \), and since \( \pi_{\lambda} \left( \pi_{\lambda}^{-1}(\mathscr{H}) \right) =\mathscr{H} \), we conclude that none of larger filter than \( \mathscr{F}_{\lambda} \) converge to \( p_{\lambda} \). Thus, \( p_{\lambda} \) is not a cluster point of \( \mathscr{F}_{\lambda} \).
% 	\qed\end{prf}

\begin{thm}\label{characterize convergence of induced filter} (Characterization of convergence of induced filter in initial topology)
	Suppose \( X \) is the topological space equipped with the initial topology induced by topological spaces \( (X_{\lambda})_{\lambda \in \Lambda} \) and maps \( (\pi_{\lambda}:X \to X_{\lambda})_{\lambda \in \Lambda} \), that is, the coarsest topology with respect to which every \( \pi_{\lambda} \) is continuous.
	Let \( (\mathscr{F}_{\lambda})_{\lambda \in \Lambda} \) be filters such that each \( \pi_{\lambda}^{-1}(\mathscr{F}_{\lambda}) \) is a filter basis.
	Let \( \mathscr{F} \) be the filter of \( X \) induced by \( (\mathscr{F}_{\lambda}) \) and \( (\pi_{\lambda}) \).
	\begin{itemize}
		\item[(1)] Each \( \mathscr{F}_{\lambda} \) converges to a point \( p_{\lambda}\) if and only if \( \mathscr{F} \) converges to \( p:=(p_{\lambda}) \).

		\item[(2)] Each \( \mathscr{F}_{\lambda} \) admits a cluster point \( p_{\lambda} \) if and only if \( \mathscr{F} \) admits a cluster point \( p \).
	\end{itemize}
\end{thm}
\begin{prf}
	(1): Sufficiency follows from continuity of \( \pi_{\lambda} \). Conversely, suppose each \( \mathscr{F}_{\lambda} \) converges to \( p_{\lambda} \).
	Let \( N \) be a neighborhood of \( p:=(p_{\lambda})_{\lambda \in \Lambda} \) in \( X \). By construction of initial topology, there are finite subset \( J \) of \( I \) such that
	\begin{equation*}
		N \supset \bigcap_{j \in J} \pi_{j}^{-1}(N_{j}),
	\end{equation*}
	where each \( N_j \) is a neighborhood of \( p_j \). There also exists \( F_j \in \mathscr{F}_j \) such that \( F_j \subset N_j \) for each \( j \in J \).
	This implies a basis of \( \mathscr{F} \) converges to \( p \) by Proposition\ref{characterize induced filter}.

	(2): Sufficiency follows again from continuity of \( \pi \). Conversely, if \( p_{\lambda} \) is a cluster point of \( \mathscr{F}_{\lambda} \), then, by Proposition\ref{characterize cluster point}, there is a filter \( \mathscr{G}_{\lambda} \) that is finer than \( \mathscr{F}_{\lambda} \) and converges to \( p_{\lambda} \) for every \( \lambda \in \Lambda \). Then (1) implies the filter \( \mathscr{G} \) induced by \( (\mathscr{G}_{\lambda}, \pi_{\lambda}) \) converges to \( (p_{\lambda}) \). Since \( \mathscr{G} \) is obviously finer than \( \mathscr{F} \), \( (p_{\lambda}) \) is a cluster point of \( \mathscr{F} \) by Proposition\ref{characterize cluster point}.
	\qed\end{prf}

\subsection{Compact Space with Filter}

\begin{prp}\label{invariance of compactness under ctn} (Invariance of compactness under continuous transformation)
	Let \( X \) be a compact space and \( Y \) be a topological space.
	If \( f:X \to Y \) is a surjective and continuous map, then \( Y \) is compact.
\end{prp}
\begin{prf}
	For any filter \( \mathscr{G} \) of \( Y \), we see that \( f^{-1}(\mathscr{G}) \) is a filter of \( X \). By assumption, \( f^{-1}(\mathscr{G}) \) admits a cluster point \( p \) in \( X \), which is, by Exercise\ref{continous mapping of cp}, also a cluster point of \( f(f^{-1}(\mathscr{G})) \supset \mathscr{G}\). This implies that \( \mathscr{G} \) also admits a cluster point, and hence \( Y \) is compact.
	\qed\end{prf}

\begin{prp}\label{invariance of compactness under closed intersection} (invariance of compactness under closed intersection)
	Every closed set of a compact space is compact.
\end{prp}
\begin{prf}
	Let \( \mathscr{F} \) be a filter of a closed subset of \( B \) of a compact space. By assumption, \( \mathscr{F} \) admits a cluster point belonging to \( B \). Thus, \( B \) is compact.
	\qed\end{prf}

\begin{prp}\label{compact -> closed if Hausdorff}
	Every compact set of a Hausdorff space is closed.
\end{prp}

\begin{prf}
	Suppose a filter \( \mathscr{F} \) of a compact set \( B \) of a Hausdorff space \( X \) converges to a point \( p \) in \( X \). By compactness, \( \mathscr{F} \) has a cluster point \( q \) in \( B \). By proposition\ref{characterize cluster point}, there is a finer filter \( \mathscr{G} \) than \( \mathscr{F} \) such that \( \mathscr{G} \to q \) in \( B \), as well as in \( X \) as a filter basis. Then Proposition\ref{characterize Hausdorff} implies \( p =q \in B \). Thus, \( B \) is closed by Exercise\ref{characterize closed}.
	\qed\end{prf}

\begin{prp}
	Let \( X \) be a compact space and \( Y \) a Hausdorff space.
	Then every continuous bijection of \( X \) into \( Y \) is homeomorphism.
\end{prp}
\begin{prf}
	It suffice to show that \( f \) is a closed mapping. Suppose \( F \) is a closed set of \( X \). Then, by by Proposition\ref{invariance of compactness under closed intersection}, it is compact, whose image under \( f \) is compact by Proposition\ref{invariance of compactness under ctn}, and thus closed by Proposition\ref{compact -> closed if Hausdorff}.
	\qed\end{prf}

\begin{thm} (Tychonoff)
	Suppose a set \( X \) is equipped with the initial topology induced by topological spaces \( (X_{\lambda})_{\lambda \in \Lambda} \) and surjective maps \( (\pi_{\lambda}: X \to X_{\lambda})_{\lambda \in \Lambda} \).
	Then \( X \) is compact if and only if each \( X_{\lambda} \) is compact.
\end{thm}
\begin{prf}
	Necessity is obvious. Conversely, suppose each \( X_{\lambda} \) is compact.
	Let \( \mathscr{F} \) be a maximal filter of \( X \). By Proposition\ref{invariace of maximality under mapping}, \( \mathscr{F}_{\lambda}:=\pi_{\lambda}(\mathscr{F}) \) is also maximal. Proposition\ref{characterize compact} implies each \( \mathscr{F}_{\lambda} \) is convergent; suppose \( \mathscr{F}_{\lambda}\to p_{\lambda} \). It then follows from Proposition\ref{characterize convergence of induced filter} that \( \prod \mathscr{F}_{\lambda} \to p:=(p_{\lambda})\), and thus \( \mathscr{F} \to p \).
	\qed\end{prf}

see\cite{bour:tplgy}.
\cite{nagata:tplgy}.

\bibliographystyle{plain}
\bibliography{ref}

\end{document}